%% Description for the finite element discretisation
%++++++++++++++++++++++++++++++++++++++
%++++++++++++++++++++++++++++++++++++++
% Description of method for exact computation of contact points
\newcommand{\pluseq}{\mathrel{+}=} %for some algorithms ...
%
\section{Methods}
A brief description of cable element, normal and tangential contact model will be added here.
 
%\paragraph{Cable element}
%For the numerical modelling of the belt, a newly developed ANCF beam element was used \cite{Pieber2021}. 
%Note that, however, for the investigations in this test case, the Eulerian axial velocity has been set to zero. A comparison to a model with Eulerian axial velocity set to the overall velocity of the belt drive, see \cite{pechstein2013}, is out of the scope of this test case, but will be performed in future investigations.
%
%
%The geometry of the ANCF beam element with reference length $L$ is defined as follows.
%If $\bar{x} \in [0,L]$ is the local coordinate of the beam element, nodal positions and slopes are given by,
%\be
%\label{eq_Sq}
%  \rv(\bar{x}, t) = \Sm(\bar{x}) \qv(t) \qquad \mbox{and} \qquad 
%  \rv'(\bar{x}, t) = \Sm' (\bar{x}) \qv(t) \eqComma
%\ee
%in which $\qv(t)$ is the vector of generalized coordinates and $\Sm$ is the matrix of shape functions defined as 
%\be
%\Sm(\bar{x})\, =\, \left[\, S_1(\bar{x})\,\Im\;\; S_2(\bar{x})\,\Im\;\; S_3(\bar{x})\,\Im\;\; S_4(\bar{x})\,\Im\, \right] \eqComma
%\ee
%in which $S_i$ are third order polynomials that can be written as,
%\bea
%\label{shapeFunctions}
%  S_1(\bar{x}) &=& 1-3\frac{\bar{x}^2}{L^2}+2\frac{\bar{x}^3}{L^3} \eqComma \quad S_2(\bar{x}) = \bar{x}-2\frac{\bar{x}^2}{L}+\frac{\bar{x}^3}{L^2} \eqComma \nonumber \\
%  S_3(\bar{x}) &=& 3\frac{\bar{x}^2}{L^2}-2\frac{\bar{x}^3}{L^3},\quad S_4(\bar{x}) = -\frac{\bar{x}^2}{L}+\frac{\bar{x}^3}{L^2} \eqComma
%\eea
%
%The description of this subsection follows the implementation of the contact element and can be found in our in-house code Exudyn \cite{2021_Gerstmayr_theDoc}.
%Text, figures and equations in this subsection are widely taken from the documentation \cite{2021_Gerstmayr_theDoc}.
%The description is extended by pseudo-codes as well as specific information that is relevant for the present test case.
%
%\begin{figure}[tbph]
%  \begin{center}
%  \includegraphics[width=12cm]{figures/ESR8_ContactFrictionCircleCable2D.pdf}
%  \end{center}
%      \caption{Sketch of cable, contact segments and circle; showing case without contact, $|\dv_{g1}| > r$, 
%               while contact occurs with $|\dv_{g1}| \le r$; the shortest distance vector $\dv_{g1}$
%               is related to segment $s_1$ (which is perpendicular to the the segment line) and 
%               $\dv_{g2}$ is the shortest distance to the end point of segment $s_2$, not being
%               perpendicular.}
%  \label{fig:ESR8_ObjectContactFrictionCircleCable2D:sketch}
%\end{figure}
%%+++++++++++++++++++++++++++++++++++++++++++++++
%\paragraph{Contact geometry}
%%
%The contact and friction between ANCF cable element and circle is realized with connector elements, each of them per circle and per cable element.
%%
%The connector represents a force element between a circle with radius $R$, which can undergo rigid body motion, and an ANCF cable element as described above.
%%The cable is discretised by splitting into a variable number of segments $s_i$, located between points $p_i$ and $p_{i+1}$. 
%The cable with reference length $L$ is discretised by splitting into $n_{cs}$ straight segments $s_i$, located between points $p_i$ and $p_{i+1}$, compare \fig{fig:ESR8_ObjectContactFrictionCircleCable2D:stickingPos}.
%%
%Note that these points can be placed at a lateral ('vertical') offset from the cable centreline.
%In order to compute the gap function for a line segment, the shortest distance of one line $s_i$ segment (with
%according point vectors $\pv_i$, $\pv_{i+1}$) to the circle centrepoint $\pv_{m0}$ is computed.
%
%With the intermediate quantities (all of them related to segment $s_i$)\footnote{we omit $s_i$ in most terms for brevity!},
%\be
%  \vv_s = \pv_{i+1} - \pv_i, \quad
%  \vv_p = \pv_{m0} - \pv_i, \quad
%  n = \vv_s\tp \vv_p, \quad \mathrm{and} \quad
%  d = \vv_s\tp \vv_s
%\ee
%and assuming that $d \neq 0$,
%%(otherwise the two segment points would be identical and the shortest distance would be $d_g = \Vert \vv_p \Vert $),
%we find the relative position $\rho$ of the shortest (projected) point on the 
%segment, which runs from 0 to 1 if lying on the segment, as
%\be
%  \rho = \frac{n}{d}
%\ee
%We distinguish three cases (see also \fig{fig:ESR8_ObjectContactFrictionCircleCable2D:sketch} for cases 1 and 2):
%    \bn
%    \item If $\rho \le 0$, the shortest distance is the distance to point $\pv_p=\pv_i$,
%    reading 
%    \be
%      d_g = \Vert \pv_{m0} - \pv_i\Vert  \quad (\rho \le 0)
%    \ee
%    \item If $\rho \ge 1$, the shortest distance is the distance to point $\pv_p=\pv_{i+1}$,
%    reading 
%    \be
%      d_g = \Vert \pv_{m0} - \pv_{i+1}\Vert  \quad (\rho \ge 1)
%    \ee
%    \item Finally, if $0 < \rho < 1$, then the shortest distance has a projected point $\pv_p$ somewhere
%    on the segment
%    \be
%      \pv_p = \pv_i + \rho \cdot \vv_s
%    \ee
%    and the distance
%    \be
%      d_g = \Vert \dv_g\Vert  = \sqrt{\vv_p\tp \vv_p - (n^2)/d}
%    \ee
%\en
%Here, the shortest distance vector for every segment results from the projected point $\pv_p$ 
%of the above mentioned cases, see also \fig{fig:ESR8_ObjectContactFrictionCircleCable2D:sketch},
%with the relation
%\be
%  \dv_g = \dv_{g,s_i}= \pv_{m0} - \pv_p \eqDot
%\ee
%The contact gap for a specific point for segment $s_i$ is defined as
%\be \label{ObjectContactFrictionCircleCable2D:gap}
%  g = g_{s_i} = d_g - r \eqDot
%\ee
%using $d_g = \Vert \dv_g\Vert $.
%    
%%++++++++++++++++++++++++++++++++++++++++++++++
%\paragraph{Contact frame and relative motion}
%%FRAME
%The contact normal vector $\nv_{s_i}$ and tangential vector $\tv_{s_i}$ are defined per segment as
%\be
%  \nv_{s_i} = \nv = [n_0, n_1]\tp = \frac{1}{\Vert \dv_{g,s_i}\Vert } \dv_{g,s_i}, \quad \tv_{s_i} = \tv = [-n_1, n_0]\tp
%\ee
%The vectors $\tv_{s_i}$ and $\nv_{s_i}$ define the local (contact) frame for further computations.
%
%The velocity at the closest point of the segment $s_i$ is interpolated using $\rho$ and computed as
%\be
%  \dot \pv_p = (1-\rho) \cdot \vv_i + \rho \cdot \vv_{i+1}
%\ee
%Alternatively, $\dot \pv_p$ could be computed from the cable element by evaluating the velocity at the contact points, but we feel that
%this choice is more consistent with the computations at position level.
%
%The gap velocity $v_n$ ($\neq \dot g$) thus reads
%\be
%  v_n = \left( \dot \pv_p - \dot \pv_{m0} \right) \nv
%\ee
%Similarly, the tangential velocity reads
%\be \label{ObjectContactFrictionCircleCable2D:vTangent}
%  v_t = \left( \dot \pv_p - \dot \pv_{m0} \right) \tv
%\ee
%In order to compute a tangential stiffness, we need a reference point on the circle.
%Thus, we continuously track the sticking position at which the cable element (or segment) and the circle 
%previously sticked together, similar as proposed by Lugr{\'i}s et al.~\cite{lugris2011b}. 
%The difference here to the latter reference, is that we explicitly exclude switching from Newton's method and that Lugr{\'i}s et al.~used
%contact points, while we use linear segments.
%%Furthermore, the bristle model goes back to a regularized version, the so-called LuGre friction model \cite{CanudasDeWitEtAl1993}, by integrating the relative velocity.
%%++++++++++++++++++++++++
%\begin{figure}[tbph]
%  \begin{center}
%  \includegraphics[width=10cm]{figures/ESR8_ContactFrictionCircleCable2DstickingPos.pdf}
%  \end{center}
%  \caption{Calculation of last sticking position; blue parts mark the sticking position calculated as $x^*_{curStick}$.}
%  \label{fig:ESR8_ObjectContactFrictionCircleCable2D:stickingPos}
%\end{figure}
%%++++++++++++++++++++++++
%
%Due to the linear segments, in comparison to the simpler case of contact points, there is the possibility to wind/unwind relative to the (last) sticking position without slipping.
%%
%Therefore, the method to compute the current or last sticking position is more involved.
%In case of sliding, 
%we compute the current sticking position, see \fig{fig:ESR8_ObjectContactFrictionCircleCable2D:stickingPos}, as the sum of the relative position at the segment $s$
%\be
%  x_{s,curStick} = \rho \cdot L_{seg}
%\ee
%in which $\rho \in [0,1]$ denotes the relative position of contact at the segment with reference length $L_{seg}=\frac{L}{n_{cs}}$.
%The relative position at the circle $c$ is
%\be
%  x_{c,curStick} = \alpha \cdot r
%\ee
%We immediately see, that under pure rolling\footnote{neglecting the effects of small penetration, usually much smaller than shown for visibility in \fig{fig:ESR8_ObjectContactFrictionCircleCable2D:stickingPos}.},
%\be
%  x_{s,curStick} + x_{c,curStick}  = \mathrm{const}.
%\ee
%Note that the vertical offset from the cable centreline, as shown in \fig{fig:ESR8_ObjectContactFrictionCircleCable2D:sketch},
%as well as the axial stretch are not considered in the calculation of (un-)winding of the segment.
%While a reduction of segment length reduces this effect because less (un-)winding occurs, a more consistent computation of this effect would require an integration of relative motion as stretch influences the local changes of the relative sticking position.
%For the same reason, the proposed example has a smaller height as compared to the original example \cite{pechstein2013}.
%
%The current sticking position $x_{curStick}$ is computed per segment as
%\be  \label{ObjectContactFrictionCircleCable2D:lastCurStick}
%  x^*_{curStick} = x_{s,curStick} + x_{c,curStick}, \quad
%\ee
%%
%Due to the possibility of switching of $\alpha+\phi$ between $-\pi$ and $\pi$, the result is normalized to
%\be \label{ObjectContactFrictionCircleCable2D:curStick}
%  x_{curStick} = x^*_{curStick} - \mathrm{floor}\left(\frac{x^*_{curStick} }{2 \pi \cdot r} + \frac{1}{2}\right) \cdot 2 \pi \cdot r, \quad
%\ee
%which gives $x_{curStick} \in [-\pi \cdot r,\pi \cdot r]$.
%The function floor() is a standardized version of rounding, available in C and Python programming languages.
%In the so-called PostNewtonStep, see below, the last sticking position is computed, $x_{lastStick} = x_{curStick}$, which is also available at the start of every time step.
%
%%++++++++++++++++++++++++++++++++++++++++++++++
%\paragraph{Contact forces: definition}
%%FORCES
%The normal contact force $f_n$ is zero for gap $g > 0$ and otherwise computed from 
%\be \label{ObjectContactFrictionCircleCable2D:contactForce}
%  f_n = k_c \cdot g + d_c \cdot v_n
%\ee
%Note that the linear spring damper model of \eq{ObjectContactFrictionCircleCable2D:contactForce} does not account for 
%nonlinear effects (increase of contact area in Hertzian contact). However, such models are common in reeving systems \cite{lugris2011b}.
%
%Friction forces are primarily based on relative (tangential) velocity at each segment.
%The linear part of the friction force is based on the relative tangential velocity $v_t$ and the velocity penalty parameter $\mu_v$ and reads
%\be
%  f_t^{(lin)} = \mu_v \cdot v_t \eqComma
%\ee
%The linear part is then used to compute a nonlinear tangential contact force.
%%
%%++++++++++++++++++++++++++++++++++++++++++++++
%\begin{figure}[tbph]
%  \begin{center}
%\pythonstyle\begin{lstlisting}
%function [q0, x0] = ImplicitStep(q, x, tol, tolPNS)
%    #This function is iteratively called at every implicit time step
%
%    #set start of step configuration to solution (q,x) of previous step
%    q_startOfStep = q
%    x_startOfStep = x
%    x0 = x
%
%    #discontinuous iteration starts here:
%    converged_PNS = False
%    while not converged_PNS:
%        q0 = q_startOfStep #reset at every PN step
%        #apply Newton method to implicit step
%        
%        converged_Newton = False
%        while not converged_Newton:
%            f = Residual(q0,x0)
%            
%            if norm(f) <= tol:
%              converged_Newton = True
%              break
%            
%            J = Jacobian(q0,x0)
%            
%            #update q0:
%            q0 = q0-inv(J)*f
%        
%        #update x:
%        [x0, errPNS] = PostNewtonStep(q0, q_startOfStep, x0, x_startOfStep)
%        
%        if errPNS <= tolPNS:
%            converged_PNS = True
%
%    #update state for next iteration
%    return [q0, x0]
%
%\end{lstlisting}
%  \end{center}
%  \caption{Pseudocode for implicit step with continuous (\texttt{q}) and discontinuous (\texttt{x}) solution variables and PostNewtonStep (PNS).
%           Note that the PostNewtonStep requires the current state (\texttt{x0}, \texttt{q0}) as well as the state at the start of the time step
%           in order to compute an appropriate update for the data variables. The tolerances \texttt{tol} resp.\ \texttt{tolPNS} are used as convergence
%           criteria for Newton resp.\ PostNewton }
%	\label{fig_ESR8_NonlinearSolver}
%\end{figure}
%
%
%\paragraph{PostNewtonStep}
%In general, a nonlinear solver, see \fig{fig_ESR8_NonlinearSolver}, is used for the computation of every implicit time step.
%Internally, a Newton method -- which requires continuity of the solution with respect to states (positions and velocities of circle and cable) --
%is applied for the solution of every nonlinear implicit time step.
%Switching may only occur outside of this iteration, which is why we introduce data variables, see below.
%Every step is started with values of the previous time step (or the initial values), while current continuous resp.\ discontinuous variables are updated in the 
%Newton resp.\ discontinuous iteration.
%
%The PostNewtonStep, which is the main update function in the discontinuous iteration, computes 3 values per segment, which are used for computation of contact forces, irrespectively of the current geometry of the contact. 
%The PostNewtonStep is called after every full Newton method and evaluates the current state w.r.t.\ the assumed data variables.
%If the assumptions do not fit, new data variables are computed.
%%This is necessary in order to avoid discontinuities in the equations, while otherwise the Newton iterations would not 
%%(or only slowly) converge.
%
%The data variables per segment are
%\be
%  [x_{gap},\, x_{isSlipStick},\, x_{lastStick}]
%\ee
%Here, $x_{gap}$ contains the gap of the segment ($\le 0$ means contact), $x_{lastStick}$ is described in 
%\eq{ObjectContactFrictionCircleCable2D:curStick}, and 
%$x_{isSlipStick}$ defines the stick or slip case,
%\bi
%  \item $x_{isSlipStick} = -2$: undefined, used for initialization
%  \item $x_{isSlipStick} = 0$: sticking
%  \item $x_{isSlipStick} = \pm 1$: slipping, sign defines slipping direction
%\ei
%    
%The basic algorithm in the PostNewtonStep, with all operations given for any segment $s_i$, can be summarized as follows:
%\bi
%  \item [I.] Evaluate gap per segment $g$ using \eq{ObjectContactFrictionCircleCable2D:gap} and store in data variable: 
%        $x_{gap} = g$
%  \item [II.] If $x_{gap} < 0$ and ($\mu_v \neq 0$ or  $\mu_k \neq 0$):
%  \bi
%    \item[II.1] Compute contact force $f_n$ according to \eq{ObjectContactFrictionCircleCable2D:contactForce}
%    \item[II.2] Compute current sticking position $x_{curStick}$ according to \eq{ObjectContactFrictionCircleCable2D:lastCurStick}
%    \item[II.3] Retrieve \texttt{startOfStep} sticking position in $x^{startOfStep}_{lastStick}$ and compute and normalize
%    difference in sticking position\footnote{in case that $x_{isSlipStick} = -2$, meaning that there is no stored sticking position, we set $\Delta x_{stick} = 0$}:
%    \be
%      \Delta x^*_{stick} = x_{curStick} - x^{startOfStep}_{lastStick}, \quad
%      \Delta x_{stick} = \Delta x^*_{stick} - \mathrm{floor}\left(\frac{\Delta x^*_{stick} }{2 \pi \cdot r} + \frac{1}{2}\right) \cdot 2 \pi \cdot r
%    \ee
%    \item[II.4] Compute linear tangential force for friction stiffness and velocity penalty: 
%      \be 
%        f_{t,lin} = \mu_v \cdot v_t + \mu_k \Delta x_{stick}
%      \ee
%    \item[II.5] Compute tangential force according to Coulomb friction model:
%    \be
%        f_t = 
%            \begin{cases} f_t^{(lin)}, \quad \quad \quad \quad \quad \quad \quad \mathrm{if} \quad 
%              \Vert f_t^{(lin)}\Vert  \le \mu \cdot \Vert f_n\Vert  \\ 
%              \mu \cdot \Vert f_n\Vert  \cdot \mathrm{Sign}(\Delta x_{stick}), \quad \mathrm{else}
%            \end{cases}          
%    \ee
%    \item[II.6] In the case of slipping, given by $\Vert f_t^{(lin)}\Vert  > \mu \cdot \Vert f_n\Vert $, we update the last sticking position in the data variable, 
%    such that the spring is pre-tensioned already,
%    \be
%      x_{lastStick} = x_{curStick} - \mathrm{Sign}(\Delta x_{stick}) \frac{\mu \cdot \Vert f_n\Vert }{\mu_k}, \quad 
%      x_{isSlipStick} = \mathrm{Sign}(\Delta x_{stick})
%    \ee
%    \item[II.7] In the case of sticking, given by $\Vert f_t^{(lin)}\Vert  \le \mu \cdot \Vert f_n\Vert $: Set $x_{isSlipStick} = 0$ and, 
%    if $x^{startOfStep}_{isSlipStick} = -2$ (undefined), we update $x_{lastStick} = x_{curStick}$, while otherwise, $x_{lastStick}$ is unchanged.
%  \ei
%  \item[III.] If $x_{gap} > 0$ or ($\mu_v = 0$ and $\mu_k = 0$), we set $x_{isSlipStick} = -2$ (undefined); this means that in the next step (if this step is accepted), there is no stored sticking position.
%  \item[IV.] Compute an error $\varepsilon_{PNS} = \varepsilon^n_{PNS}+\varepsilon^t_{PNS}$,
%              with physical units forces (per segment point), for \texttt{PostNewtonStep}:
%  \bi
%    \item[IV.1] if gap $x_{gap,lastPNS}$ of previous \texttt{PostNewtonStep} had a different sign as the current gap, set
%    \be
%      \varepsilon^n_{PNS} = k_c \cdot \Vert x_{gap} - x_{gap,lastPNS}\Vert
%    \ee
%    \item[] while otherwise $\varepsilon^n_{PNS}=0$.
%    \item[IV.2] if stick-slip-state $x_{isSlipStick,lastPNS}$ of previous \texttt{PostNewtonStep} is different from current $x_{isSlipStick}$, set
%    \be
%      \varepsilon^t_{PNS} = \Vert \left(\Vert f_t^{(lin)} \Vert  - \mu \cdot \Vert f_n\Vert  \right)\Vert 
%    \ee
%    while otherwise $\varepsilon^t_{PNS}=0$.
%  \ei
%\ei
%
%    \paragraph{Computation of connector forces in Newton}
%    %
%    The computation of the forces produced by the contact-friction element is done during Newton iterations.
%    For efficiency, the force computation is only performed if the \texttt{PostNewtonStep} determined contact in any segment.
%
%    The following operations are performed for each segment $s_i$, if 
%    $x_{gap, s_i} <= 0$:
%    \bi
%      \item[I.] Compute contact force $f_n$, \eq{ObjectContactFrictionCircleCable2D:contactForce}.
%      \item[II.] In case of sticking:
%      \bi
%        \item [II.1] the current sticking position $x_{curStick}$ is computed from \eq{ObjectContactFrictionCircleCable2D:lastCurStick}, and the difference of current and last sticking position reads\footnote{see the difference to the \texttt{PostNewtonStep}: we use $x_{lastStick}$ here, not the \texttt{startOfStep} variant.}:
%        \be
%          \Delta x^*_{stick} = x_{curStick} - x_{lastStick}, \quad
%          \Delta x_{stick} = \Delta x^*_{stick} - \mathrm{floor}\left(\frac{\Delta x^*_{stick} }{2 \pi \cdot r} + \frac{1}{2}\right) \cdot 2 \pi \cdot r
%        \ee
%        \item [II.2] however, if the friction stiffness $\mu_k=0$ or if $x_{isSlipStick} = -2$, we also set $\Delta x_{stick}=0$
%        \item [II.3] using the tangential velocity from \eq{ObjectContactFrictionCircleCable2D:vTangent}, the linear tangent force follows as
%        \be
%          f_{t,lin} = \mu_v \cdot v_t + \mu_k \Delta x_{stick}
%        \ee
%        \item [II.4] the tangential friction force then results in\footnote{see again difference to \texttt{PostNewtonStep}!},
%        \be
%            f_t = 
%                \begin{cases} f_t^{(lin)}, \quad \quad \quad \quad \quad \quad \quad \mathrm{if} \quad 
%                  \Vert x_{isSlipStick} \Vert \neq 1 \\ 
%                  \mu \cdot \Vert f_n\Vert  \cdot x_{isSlipStick}, \quad \mathrm{else}
%                \end{cases}
%        \ee 
%      \ei
%    \ei
%    %++++++++++++++++++++++++++++++++++++++++++++++
%    \paragraph{Computation of generalized forces for circle and ANCF element}
%    Contact forces $\fv_i$ with $i \in [0,n_{cs}]$ are applied at the points $p_i$, and they are computed for every contact segment.
%    For every contact computation, first all contact forces at segment points are set to zero. 
%    In general, we may distinguish two cases, compare \fig{fig:ESR8_ObjectContactFrictionCircleCable2D:normals}.
%    %++++++++++++++++++++++++
%    \begin{figure}[tbph]
%      \begin{center}
%      \includegraphics[width=16cm]{figures/ESR8_ContactFrictionCircleCable2Dnormals.pdf}
%      \end{center}
%      \caption{Choice of normals and tangent vectors for calculation of normal contact forces and tangential (friction) forces; 
%      left: case SN with \texttt{usePointWiseNormals=False};
%      right: case PWN with \texttt{usePointWiseNormals=True} }
%    	\label{fig:ESR8_ObjectContactFrictionCircleCable2D:normals}
%    \end{figure}
%    %++++++++++++++++++++++++
%    Segment normals ($=\,$SN) always lead to good approximations for normal directions, irrespectively of short or extremely long segments as compared to the circle. However, in case of segments that are short as compared to the circle radius, normals computed from the centre of the circle to the segment points ($=\,$PWN) are more consistent and produce tangents only in circumferential direction, which may improve behaviour in some applications. Segment normals are used in the investigations here, reading:
%    \bi
%    \item[] \mybold{CASE SN}: use \mybold{S}egment \mybold{N}ormals\\
%    If there is contact in a segment $s_i$, i.e., gap state $x_{gap} \le 0$, see \fig{fig:ESR8_ObjectContactFrictionCircleCable2D:sketch} (right), contact forces $\fv_{s_i}$ are computed per segment,
%    \be
%      \fv_{s_i} = f_n \cdot \nv_{s_i} + f_t \tv_{s_i}
%    \ee
%    and added to every force $\fv_i$, initialized with zero, at segment points according to
%      \bea
%        \fv_i &=& \fv_i + (1-\rho) \cdot \fv_{s_i}      \\ \nonumber
%        \fv_{i+1} &=& \fv_{i+1}  + \rho \cdot \fv_{s_i}
%      \eea
%    while in case $x_{gap}  > 0$ nothing is added.
%    \ei
%    The forces $\fv_i$ are then applied to the ANCF cable element at according segment points.
%    The forces on the circle (pulley) are computed as the total sum of all
%    segment contact forces, 
%    \be
%      \fv_{m0} = -\sum_{s_i} \fv_{s_i} 
%    \ee
%    and additional torques on the circle's rotation simply follow from
%    \be
%      \tau_{m0} = -\sum_{s_i} r \cdot f_{t_{s_i}} \eqDot
%    \ee
%   