\section{Description of code}
For simulating the system we are using the multibody dynamics code Exudyn \cite{Gerstmayr2022}.
%
The code in divided into sections (1, 2,…, 8) and subsections (A, B, …) for easier documenting and processing, see section 3 with subsections (A, B, ..., E):
\bi 
\item{In the first section we import necessary modules.}
\item{The third section consists of the Parameter Function. This function will be repeatedly called from Parameter Variation to update the value of the variables for which we perform variations.} 
\item{In subsection 3.A we create a class P which contains all parameters for which we can perform Parameter variations. First the parameters are given their default values. Then through:
\pythonstyle
\begin{lstlisting}
	for key,value in parameterSet.items():
		setattr(P,key,value)
\end{lstlisting}		
we updated the varying parameters.}
\item{In subsection 3.B we create the model which is going to be updated for every parameter variation.}
%
%
\item{For the ANCF beam elements modeling the belt we are using ObjectANCFCable2D, see the documentation of Exudyn\footnote{EXUDYN V1.3.86.dev1; https://github.com/jgerstmayr/EXUDYN}, theDoc.
Parameters used for ObjectANCFCable2D are defined in Table \ref{tab:ObjectANCFCable2D}.}
\item{For increasing the angular velocity, we are using a user function
\pythonstyle
\begin{lstlisting}
def UFvelocityDrive(mbs, t, itemNumber, lOffset): 
	if t < tAccStart:  # driving start time
		v = 0
	if t >= tAccStart and t < tAccEnd:
		v = omegaFinal/(tAccEnd-tAccStart)*(t-tAccStart)
	elif t >= tAccEnd:
		v = omegaFinal
	return v
\end{lstlisting}                
}
%
\begin{table}
    \caption{Parameters for ObjectANCFCable2D} \label{tab:ObjectANCFCable2D}
    \centering
    %\begin{tabular}{@{}lrlp{0.4\textwidth}@{}} \toprule
    \begin{tabular}{c|c} \hline
        Parameter & Value \\ \hline 
        physicsMassPerLength & $\rho A$\\
        physicsBendingStiffness & $EI$\\
        physicsAxialStiffness & $EA$\\
        physicsBendingDamping & $dEI$\\
        physicsAxialDamping & $dEA$\\
        physicsReferenceAxialStrain  & $\varepsilon_{ref}$\\
        physicsReferenceCurvature & $0$\\
        useReducedOrderIntegration & $2$\\
        strainIsRelativeToReference & False\\
        visualization & (to be added...)\\ \hline
        %\bottomrule
    \end{tabular}
\end{table}
\ei
(The file is still under construction.)
\section{Installation and running}
\subsection{Installing python and Exudyn}
The code was tested in a Windows pc using Anaconda, 64bit, Python 3.7.6 and Spyder 4.0.1 which is included in the Anaconda installation.

For installing Exudyn PIP INSTALLER (pypi.org) was used based on the following instructions:
%In order to install Exudyn using pip installer:

Pre-built versions of Exudyn are hosted on \texttt{pypi.org}, see the project
\bi
 \item \exuUrl{https://pypi.org/project/exudyn}{https://pypi.org/project/exudyn}
\ei
As with most other packages, in the regular case (if your binary has been pre-built) you just need to do\footnote{If the index of pypi is not updated, it may help to use \texttt{pip install -i https://pypi.org/project/ exudyn} }
\bi
  \item[] \texttt{pip install exudyn}
\ei
On Linux (currently only pre-built for UBUNTU, but should work on many other linux platforms), {\bf update pip to at least 20.3} and use 
\bi
  \item[] \texttt{pip3 install exudyn}
\ei
For pre-releases (use with care!), add '$--$pre' flag:
\bi
  \item[] \texttt{pip install exudyn $--$pre}
\ei                 
For more information for installing Exudyn see the theDoc\footnote{EXUDYN V1.3.86.dev1; https://github.com/jgerstmayr/EXUDYN}.
%What exactly should I cite here? How to refer to the same footnote?
\subsection{Running the code}
In section 7 of the code one can choose between performing single simulation and performing parameter variation.
An option for plotting figures is given. Solutions for parameters variations were stored in solution folder. Solutions from new runs are stored by default in solutionNosync. %How to use underscore in the text?


(The file is still under construction.)